\section{Installationsanleitung}

\subsection{Benötigte Programme und Voraussetzungen}
Für dieses Projekt wird zusätzliche Software benötigt:
\begin{itemize}
\item Docker \footnote{https://www.docker.com/}
\item Visual Studio \footnote{https://visualstudio.microsoft.com/de/}
\item MySql-Connector (optional, nur falls notwendig) \footnote{https://dev.mysql.com/get/Downloads/Connector-Net/mysql-connector-net-8.0.13.msi}
\end{itemize}
\raggedright

Um die Datenbank zu erstellen muss Docker gestartet sein. Die Datenbank kann mit dem \textit{Powershell-Script} \grqq{}run.ps1 \grqq{} automatisch generiert werden.
Falls dieses Script wegen fehlender Berechtigungen nicht ausgeführt werden kann, muss eine \textit{Shell} (Git-Bash oder ähnliches) im Ordner mit der Docker Compose Datei \grqq{}docker-compose.yaml \grqq{} geöffnet und folgenden Befehle nacheinander ausgeführt werden:
\begin{minted}{dockerfile}
docker stop $(docker ps -a -q)
docker rm $(docker ps -a -q)
docker-compose up --build --force-recreate
\end{minted}

\subsection{Datenbank}
Für die Datenbank muss die SQL Datei \grqq{}create-wetr.sql \grqq{} im Ordner sql/Create ausgeführt werden. Falls das Unit Testing auch ausgeführt werden soll, wird die zweite SQL Datei \grqq{}create\textunderscore wetr-unit.testing.sql \grqq{} auch benötigt.
\newline\newline
Für die Füllung der Datenbank kann die SQL Datei \grqq{}InsertEverythingWithoutMeasurement.sql\grqq{} verwendet werden. Für die Measurement Daten muss spezieller vorgegangen werden dadurch kann aber die Insert Zeit drastisch verringert werden:
\begin{itemize}
\item Wetr.Generator.exe im Ordner sql/Insert starten
\item Die resultierenden \grqq{}.bulk\grqq{} Dateien in den sql Ordner kopieren (der phpMyAdmin Container wurde mit dem sql Ordner verbunden um diese Dateien verwenden zu können)
\item http://localhost:8080/ im Browser aufmachen \newline (phpMyAdmin sollte starten)
\item Benutzer: root Passwort: 0c1cd84e
\item Datenbank wetr auswählen
\item Zum SQL Tab wechseln
\item Die nachfolgenden SQL Anweisungen eingeben und ausführen
\end{itemize}
\newpage
\begin{minted}{sql}
LOAD DATA LOCAL INFILE "/tmp/sql/insert/measurementsDownfall.bulk" INTO TABLE measurement
FIELDS TERMINATED BY ', ' ENCLOSED BY "'"
LINES TERMINATED BY '\r\n';

LOAD DATA LOCAL INFILE "/tmp/sql/insert/measurementsHumidity.bulk" INTO TABLE measurement
FIELDS TERMINATED BY ', ' ENCLOSED BY "'"
LINES TERMINATED BY '\r\n';

LOAD DATA LOCAL INFILE "/tmp/sql/insert/measurementsTemperature.bulk" INTO TABLE measurement
FIELDS TERMINATED BY ', ' ENCLOSED BY "'"
LINES TERMINATED BY '\r\n';

LOAD DATA LOCAL INFILE "/tmp/sql/insert/measurementsWind.bulk" INTO TABLE measurement
FIELDS TERMINATED BY ', ' ENCLOSED BY "'"
LINES TERMINATED BY '\r\n';

LOAD DATA LOCAL INFILE "/tmp/sql/insert/measurementsWindDirection.bulk" INTO TABLE measurement
FIELDS TERMINATED BY ', ' ENCLOSED BY "'"
LINES TERMINATED BY '\r\n';

LOAD DATA LOCAL INFILE "/tmp/sql/insert/measurementsPreassure.bulk" INTO TABLE measurement
FIELDS TERMINATED BY ', ' ENCLOSED BY "'"
LINES TERMINATED BY '\r\n';
\end{minted}